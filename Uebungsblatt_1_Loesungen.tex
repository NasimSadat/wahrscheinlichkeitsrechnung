
\documentclass{article}
\usepackage[utf8]{inputenc}
\usepackage{amsmath}
\usepackage{amssymb}
\usepackage{geometry}
\geometry{a4paper}

\title{Lösungen zum Übungsblatt 1}
\author{}
\date{Sommersemester 2024}

\begin{document}

\maketitle

\section*{Aufgabe P1}
\subsection*{a) Ergebnisraum}
Geeigneter Ergebnisraum für das Werfen zweier nicht unterscheidbarer Würfel ist $\{ (i, j) : 1 \leq i \leq j \leq 6 \}$. Dieser berücksichtigt, dass die Würfel nicht unterscheidbar sind und vermeidet doppelte Kombinationen wie (1,2) und (2,1).

\subsection*{b) Laplace-Annahme}
Die Laplace-Annahme ist sinnvoll, da jeder Wurf gleich wahrscheinlich ist. Da die Würfel fair sind, hat jedes mögliche Paar (i, j) die gleiche Wahrscheinlichkeit.

\subsection*{c) Ereignisse}
\begin{itemize}
    \item $A1 = E \cap F$: Das Ereignis, dass die Augensumme ungerade ist und mindestens einer der Würfel eine 3 zeigt.
    \item $A2 = F \cup G$: Mindestens einer der Würfel zeigt eine 3 oder die Augensumme ist 7.
    \item $A3 = (E \cup F) \cap G$: Die Augensumme ist 7 und zusätzlich ist entweder die Augensumme ungerade oder mindestens ein Würfel zeigt eine 3.
    \item $A4 = E \cup (F \cap G)$: Die Augensumme ist ungerade oder mindestens ein Würfel zeigt eine 3 und die Augensumme ist 7.
    \item $A5 = G \setminus E$: Die Augensumme ist 7, aber nicht ungerade.
    \item $A6 = E \cap F \cap G$: Die Augensumme ist ungerade, mindestens einer der Würfel zeigt eine 3, und die Augensumme ist 7.
\end{itemize}

\section*{Aufgabe P2}
Beweise und grafische Darstellungen der Mengengleichheiten.

\section*{Aufgabe H1}
Mengen im Ergebnisraum $\Omega = [-1, 1]^2$.
\begin{itemize}
    \item $A$: $\{(x, y) \in \Omega : x + y \geq 0\}$
    \item $A \cap B$: $\{(x, y) \in \Omega : x + y \geq 0 \text{ und } |x| + |y| \geq 1\}$
    \item $B \cup C$: $\{(x, y) \in \Omega : |x| + |y| \geq 1 \text{ oder } x^2 \geq y\}$
\end{itemize}

\section*{Aufgabe H2}
Beweis der Gesetze von de Morgan unter Verwendung einer beliebigen Indexmenge $I$.
\begin{align*}
    \bigcup_{i \in I} A_i &= \bigcap_{i \in I} \overline{A_i} \\
    \bigcap_{i \in I} A_i &= \bigcup_{i \in I} \overline{A_i}
\end{align*}

\end{document}
